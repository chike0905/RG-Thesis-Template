\chapter{背景}
\label{background}

本章では本研究の背景となる,論文とはなんなのか,およびその構成要素について述べる.

\section{論文とは}
\label{background:whtthesis}
論文とは,人類に新たな発見・知見をもたらす文書である.
既存の何か問題を解決する新たな発見があり,それの解法が正しいことを``論理的''に示す.
多くの場合,``査読''と呼ばれるプロセスを経て,著者以外に正当性を検証される\footnote{学会論文には査読を経ずに,投稿すれば発表ができ,学会論文集(Proceedings)に掲載されるものもある.} .
表~\ref{table:typepaper}に論文の大まかな分類を示す.
それぞれにおいて,英語で執筆され,全世界的に公開されるものを``国際論文'',日本語で執筆されるものを``国内論文''と呼ぶ.
本稿の想定読者\footnote{RGの学部生}は,少なくとも学位論文を書かなければ卒業することはできない.

\begin{table}[!hbtp]
    \begin{center}
        \caption{論文の分類}
				\label{table:typepaper}
  			\begin{tabular}{|l|l|}
					\hline
    			学位論文 & 学位(学士・修士・博士)を取得するための論文 \\
					\hline
    			ジャーナル(論文誌)論文 & 論文を掲載する雑誌に掲載される論文  \\
    			\hline
					学会論文 & 学会で発表するための論文 \\
    			\hline \hline
					プレプリント & ArXivなどで査読を経ずに公開される論文  \\
  				\hline
				\end{tabular}
		\end{center}
\end{table}

多くの学部生は論文を書いた経験はないと考えられ,そうした人はどのように論文を執筆したら良いのだろうか.
少なくとも学部生レベルで``人類に全く新しい知見''を切り開くのは困難である.
世界中では優秀な研究者たちが日進月歩で研究をしている.
その研究者たちが全く考えが及ばない内容を,一個人が突然発見することはほとんどないと言っていいだろう.

論文を書くにあたり,最も重要な最初のステップは,``論文を読むこと(サーベイ)''である.
先人たちがいくら素晴らしいシステムを考案していても,全ての問題が解決され,完璧なシステムを作ることは困難である\footnote{``銀の弾丸は存在しない''などと言われる.}.
後述するが,論文ではその論文で未解決な問題を列挙するような節を作る.
我々が最初のステップとして実施できることは,既存の論文を読み,今までの先人たちが解決できなかった問題を探ることである.

そうして見つかった問題に対して,少しでも解決方法が提示できれば,学部生としては上出来と言わざるを得ない.
むしろ,今までの世界中で行われている研究の流れに沿って,未解決な問題を解決できたのならば,それほど素晴らしいことはない.
こうして今までの研究の上に,自分の研究を行うことで新たな知見を開くことを``巨人の肩に立つ(nani gigantum umeris insidentes)''と言う.
自分がいきなり巨人になることは難しいが,巨人の肩に乗ればほんの少しだけ,巨人よりも上の世界を見ることができるのだ.

\if0
\begin{figure}[h]
    \begin{center}
        \includegraphics[scale=0.4]{./img/hashrate.png}
        \caption{2017年1月のハッシュレート分布 出典:Blockchain.info\cite{bitcoinhashrate}}
        \label{img:hashrate}
    \end{center}
\end{figure}
\fi
