\chapter{背景}
\label{background}

本章では本研究の背景となる,論文とはなんなのか,およびその構成要素について述べる.

\section{論文とは}
\label{background:whtthesis}
論文とは,人類に新たな発見・知見をもたらす文書である.
既存の何か問題を解決する新たな発見があり,それの解法が正しいことを``論理的''に示す.
多くの場合,``査読''と呼ばれるプロセスを経て,著者以外に正当性を検証される\footnote{学会論文には査読を経ずに,投稿すれば発表ができ,学会論文集(Proceedings)に掲載されるものもある.} .
表~\ref{table:typepaper}に論文の大まかな分類を示す.
それぞれにおいて,英語で執筆され,全世界的に公開されるものを``国際論文'',日本語で執筆されるものを``国内論文''と呼ぶ.
本稿の想定読者\footnote{RGの学部生}は,少なくとも学位論文を書かなければ卒業することはできない.

\begin{table}[!hbtp]
    \begin{center}
        \caption{論文の分類}
				\label{table:typepaper}
  			\begin{tabular}{|l|l|}
					\hline
    			学位論文 & 学位(学士・修士・博士)を取得するための論文 \\
					\hline
    			ジャーナル(論文誌)論文 & 論文を掲載する雑誌に掲載される論文  \\
    			\hline
					学会論文 & 学会で発表するための論文 \\
    			\hline \hline
					プレプリント & ArXivなどで査読を経ずに公開される論文  \\
  				\hline
				\end{tabular}
		\end{center}
\end{table}

多くの学部生は論文を書いた経験はないと考えられ,そうした人はどのように論文を執筆したら良いのだろうか.
少なくとも学部生レベルで``人類に全く新しい知見''を切り開くのは困難である.
世界中では優秀な研究者たちが日進月歩で研究をしている.
その研究者たちが全く考えが及ばない内容を,一個人が突然発見することはほとんどないと言っていいだろう.

論文を書くにあたり,最も重要な最初のステップは,``論文を読むこと(サーベイ)''である.
先人たちがいくら素晴らしいシステムを考案していても,全ての問題が解決され,完璧なシステムを作ることは困難である\footnote{``銀の弾丸は存在しない''などと言われる.}.
後述するが,論文ではその論文で未解決な問題を列挙するような節を作る.
我々が最初のステップとして実施できることは,既存の論文を読み,今までの先人たちが解決できなかった問題を探ることである.

そうして見つかった問題に対して,少しでも解決方法が提示できれば,学部生としては上出来と言わざるを得ない.
むしろ,今までの世界中で行われている研究の流れに沿って,未解決な問題を解決できたのならば,それほど素晴らしいことはない.
こうして今までの研究の上に,自分の研究を行うことで新たな知見を開くことを``巨人の肩に立つ(nani gigantum umeris insidentes)''と言う.
自分がいきなり巨人になることは難しいが,巨人の肩に乗ればほんの少しだけ,巨人よりも上の世界を見ることができるのだ.

\section{論文の構成}
\label{background:structure}
本節では,一般的なシステム系論文の章立てと,その内容にどんなものを書くべきかを概説する.
大まかな論文の流れ(ストーリ)は,まず本研究が取り組む分野の``現状を整理''をして,現状で未達成な``問題点の指摘``をした上で,``解決策の提示''を行う.
その上で,解決策が正しく問題が``解決できていることの証明''を行い,その問題が解決できたことによって``どのようなことが可能になったか''を明らかにする.
この流れを示すためには,概ね章立ての形式は定式化されたものになる.
表~\ref{table:structurepaper}に,基本的なシステム系論文の章立てと,その内容を示す.

\begin{table}[!hbtp]
    \begin{center}
        \caption{システム系論文の章立て例}
				\label{table:structurepaper}
  			\begin{tabular}{|l|l|}
					\hline
          章立て & 概要 \\
					\hline
          概要 / Abstract & (章立てとは独立している)論文全体の要約 \\
					\hline
          序章 / Introduction & 取り組む分野全体と,本研究の目的と手法の概観\\
					\hline
          背景 / Background & この論文を理解するにあたっての背景知識\\
					\hline
          問題 / Problem & 背景を踏まえた取り組む課題と、その解決要件の整理 \\
					\hline
          提案手法 / Proposed Method & どのような方法で解決要件を満たすのか \\
					\hline
          実装 / Implementation & 実際にシステムを構築した詳細 \\
					\hline
          評価 / Evaluation & 提案手法が解決要件を満たしていることの検証 \\
					\hline
          関連研究 / Related Works & 問題に対する既存の取り組みとこの論文との差異 \\
					\hline
          結論 / Conclusion & 何を問題として、何を提案し、評価した結果の概観\\
  				\hline
					\hline
          謝辞 / Acknowledgment & この論文を書くにあたってお世話になった方への感謝\\
  				\hline
				\end{tabular}
		\end{center}
\end{table}

\subsection{概要}
概要は,この論文の全体の要約を示し,基本的に概要さえ読めば何をした論文かがわかるようになっている.
この章の執筆にあたり,ファカルティのRodさんのブログがとても参考になる~\cite{rodblog}.
以下に一部引用する.
\begin{quote}
  6 sentences in an abstract:\\
  1. Define the problem you're solving\\
  2. Give the key idea for how you solved it\\
  3. Describe how you demonstrate the success of your solution\\
  4. Give key results, preferably numerically\\
  5. Describe how this impacts the world/industry/whatever (big picture)\\
\end{quote}
まずは,どんな問題を解こうとするのかを定義し,それに対してどのように解くのか,を書く.
その後,その解決策がなぜ成功するのかを示すための実験内容,その重要な結果を述べる.
最後に,その結果によって世界に対してどのような貢献,良いことがあるのかを述べる.
以上の項目が,概要には書かれるべき内容である.

Rodさんのブログでは,先述の内容を``6文''で述べるべきと言っている.
後ほど論文の読み方でも触れるが,概要は読者にとってその論文を読むべきかどうかの判断材料として最も重要な節である.
そのため,概要はなるべく簡潔かつ論文の面白さが伝わるようになっていなければならない.
卒業論文の概要は必ずしも6文である必要はないが,冗長にならないように,かつ``簡潔''であるかどうかは意識するべきである.

\subsection{序論}
序論では,論文の導入として,この論文が何の分野で何を目的に,何をしようとしているのか,を概観する.
その中で,現状の研究ではどこまで実現されているが,限界がどこにあり,この論文はそれを解決しようとしている,ということを明白にする必要がある.
序論は,概要を読んだ人が次に読む部分であり,研究の方向性を理解するためにもっとも重要な章である.
そのため,この章を読んでいるときに,初見の読者に対して専門用語などが唐突に出てくることなく、理解できるように書かれていなければならない~\footnote{論文一般として,専門用語は唐突に出してはいけない.論文の読者はこの分野について無知であることを前提とし,丁寧に先頭から読んでいくと全て理解できるように論文は書かなければならない}.

序論では,何をしようとしているのか(目的),そのためにこの論文では何をしているのか(手法)を書かなければならない.
そこで注意しなければならないのは,目的と手法を混同しないことである.
例えば,``動画配信のための高速な通信プロトコル''を提案する研究について考える.
ここでは,``高速な通信プロトコル''を作ることは,動画を配信するために遅延などが発生してはいけないために必要な手法であって,目的ではない.
何かをしようとするために,何かの新たな技術を生み出すのであって,技術を作ること自体が目的ではないことに注意しなければならない.


\subsection{背景}
\subsection{問題}
\subsection{提案手法}
\subsection{実装}
\subsection{関連研究}
\subsection{結論}
\subsection{謝辞}

\section{論文を執筆するプロセス}
\label{background:process}
WIP
\subsection{サーベイの方法}
\label{backgorund:servey}
これまでに出ている論文を探し,自分が興味ある分野の最先端の研究と現状の課題を整理する作業を``サーベイ''と呼ぶ.
サーベイにおいては,むやみに論文を読むのではなく,その分野において重要な論文を見つけ出し,そこから派生する研究を紐解いていくことが重要である.
この作業によって,世界中でどんな研究がこれまでどんな経緯で行われてきたかということが明らかになっていく.
その上で,自分がどんな問題に対して,どんな解決策を提示し,それによってどのような良いことが生まれるのか(貢献),を整理していくことが研究に他ならない.

まず,論文を探すときはGoogle Scholer~\cite{googlescholer}などの論文検索サービスを利用するのが良いだろう~\footnote{その他探してきた論文をまとめるMendeley~\cite{mendeley}などの文献管理サービスを利用するとより今後捗るが,ここでは紹介のみに留める.}.
ひとまず自分の興味のある分野のキーワードを調べてみよう.
図~\ref{img:scholersearch}は``Bitcoin''で検索した結果である.
デフォルトではキーワードに対して関連性の高い論文順に並んでいる.

\begin{figure}[htb]
    \begin{center}
        \includegraphics[width=450pt]{./img/scholersearch.png}
        \caption{Google Scholer検索結果画面}
        \label{img:scholersearch}
    \end{center}
\end{figure}

どの論文を読むべきかの一つの指標となるのが,被引用数である.
~\ref{background:whtthesis}節で述べたように,多くの論文は既存の論文を``引用''し,新規の貢献を述べている.
ここで,多くの論文に引用されるということは,その論文は``比較的''多くの研究がその上で実施できるレベルの,その分野を切り開いた論文である可能性が高い.
図~\ref{img:scholersearch}でも,Bitcoinの最初の設計文書である``Bitcoin: A peer-to-peer electronic cash system''が比較的高い引用数を示している.
一方で,被引用数が高くても,必ずしも論文としての質が高いとは限らない場合もある.
被引用数は一つの指標であり,傾向を示すものでしかないことに注意するべきである.

また,被引用数が高いだけでなく当然内容についても精査しなければならない.
その場合は,タイトルを眺め,興味のありそうなものをとりあえず読むのが良いだろう.
しかし,ここでいきなり論文の本文を読んではいけない.
論文には必ず``概要(abstract)''の節が存在し,その論文の要約が掲載されている.
要約を読み,中身を把握できたら初めて論文本文を読むことになる.

\subsection{論文の読み方}
\label{background:readpaper}
論文の本文を読む時,必ずしも先頭から順に読む必要はない.
論文は,必ず何かしらの``解きたい問題''があり,それに対する``解法''を提示するものである.
その``何をどう解いたか''の核となる部分がわかれば,必ずしも全て詳細に読まなくても良い\footnote{自分の研究に対して,重要な先行研究である場合は,それとの差異をはっきり述べるためにも詳細に読む必要はある.ここで言っているのは,サーベイや普段読んでいく中で,全ての論文を詳細に読む必要はない,ということである.}.





\if0
\begin{figure}[h]
    \begin{center}
        \includegraphics[scale=0.4]{./img/hashrate.png}
        \caption{2017年1月のハッシュレート分布 出典:Blockchain.info\cite{bitcoinhashrate}}
        \label{img:hashrate}
    \end{center}
\end{figure}
\fi
